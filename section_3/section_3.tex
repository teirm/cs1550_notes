\documentclass[../base_file/cs1550_notes.tex]{subfiles}

\begin{document}
\chapter{Processes}
\section{Handling Processes}
\textbf{Multiprogamming} - a single program will not saturate a CPU\@.  The OS needs
to decide which process to run.  The CPU seems to only get a continuous stream of
instructions\@. INSERT FIGURE\@.
Every time we stop a process we introduce work.  We can do this work during I/O.\\

\textbf{RAM does not get faster with more transistors}.  The improvement is linear
in speed; however, RAM \textbf{capacity grows exponentially}.\\

\textbf{Ahmdahl's Law} -- Diminishing returns in increasing speed.  I/O is idle process
time.\\

The hope is to interleave the waits of a process with the runs of another.\\

\textbf{Process} - a running program and its associated data.\\

\textbf{Ready} - A process is ready if it has everything it needs to run except CPU time.\\

\textbf{Pseudo-Parallelism} - Juggling processes.  Blocked means you are just a choice for
scheduling.\\

In a process life-cycle, a process can leave ready state by calling \textbf{exit()}.  Processor
time is fast in compared to I/O time.  OS can block (A program is blocked wiating for I/O).  Since
the program is not ready, it is not a canidate for scheduling.  OS can instead run another process.
Hardware interrupt will tell OS that a blocked process got resources.  The process then becomes ready.\\

\textbf{Batch System} - The only way to stop a process is to exit().\\

If you choose to block, you are vulnerable to programs taht neither exit or block.  Consider the following:
\textit{jump: jmp jump} (infinite loop that does nothing).  Greedy process with CPU time.\\

\textbf{Remember: The OS is not a ready process that is scheduled}.  A greedy process starves even the OS
from CPU time, but this is tied to the greedy process.\\

One solution is to create an even in the long program -- a \textbf{Syscall}.  Asumes programmer did this
(\textbf{yield()} syscall for example).  This gives us a \textbf{Cooperative multitasking system} which 
is not the ultimate goal.  A lot depends on how the process is written.\\

We can't do this in code -- do this in \textbf{hardware internal clock}\@.  A hardware interrupt due to
a time.\\

Taking a way a resource is called premption. This gives us a \textbf{Preemptive Multitasking System}.

\section{How to choose a process}
Table or Linked List containing:
	\begin{multicols}{3}
	\begin{itemize}
	\item Process Management
   		\begin{itemize}
		\item State
		\item Priority
		\item PID
		\item PPID
		\item signal handlers
		\item stats (start time/total CPU usage)
		\end{itemize}
	\columnbreak
	\item File Management
		\begin{itemize}
		\item File descriptor
		\item Root directory
		\item Current Working Directory
		\item UID
		\item GID
		\end{itemize}
	\columnbreak
	\item Memory Management
		\begin{itemize}
		\item Page table pointer
		\item Pointers to text on text stack data
		\end{itemize}
	\end{itemize}
	\end{multicols}
\textbf{Thread} -- a stream of instructions and associated state\@.  A thread is different than a process if 
we have more than one of them\@. Tasks can communicate between multiple separate threads (different processes)
via traditional I/O OS\@.

\section{User Threads and Kernel Threads} 
pthreads abstracts system threading implementation.\\
\textbf{Kernel Threading} -- OS knows about threads.\\
\textbf{User Threading} -- OS knows nothing about threads and only has a process table.\\
User threading depends upon library -- \textbf{This is not pthreads}\@. Pthreads exists
regardless of user/kernel threading.\\

If kernel threadign: pthread.create() is a syscall.
If user threading: library that is linked against\\

Kernel threading costs context switches -- therefore user threading may be faster.\\

\textbf{Synchronization is not the problem right now}\\

Issues with preemption:
	\begin{itemize}
	\item in kernel threading we can switch to a new thread or process (schedule)
	\item in user threading schedule can only choose between between processesses
			(the greedy thread does not know of preemption -- kernel would have to do it 
			 and kernel can't see it).  A processes can yield().  It only hurts one 
			process (The one containing the greedy thread)
	\item in kernel threading pthread.yield() maybe a NOOP
	\end{itemize}

In user threading we can customize scheduling decisions.\\

 User Threading - a thread calls scanf and the whole process computation is blocked.
 However, in kernel threading other threads can keep working.\\

There is the nonblocking I/O or syscall.  It immediately returns and more than likely
your data is not ready.\\

In the user threading library, scanf can make a different syscall of the nonblocking
type.  The behavior of this syscall is along the lines of "try again or try again and 
I'll get it.  This can be in the library.  - the thread table is there so we can 
simulate "blocking" and switch to a different thread.\\

Examples:\\
Select() on a file descriptor, checks if the data is ready.\\
A videogame is drawing state with select and read, in a loop.\\

Select is a unix syscall. In linux there is poll() and epoll().\\
Between User threading and Kernel threading try to hybridize.  \textbf{Scheduler Activation}
(upcall) is pure hybrid because the OS is told what to do by process.\\

You need to have kernel threading and link user threading on top (Another hybrid aproach).\\
More Syscalls? Linux 2.6 had faster kernel threading than Linux 2.4 -- due to a bad
implementation.
\end{document}

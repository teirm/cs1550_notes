\documentclass[../base_file/cs1550_notes.tex]{subfiles}

\begin{document}
\chapter{Overview}
A brief overview of the course.
\section{Operating Systems Manage Resources and Abstract Details}
\textbf{Resources: CPU Time, Memory, I/O Devices, Security}
	\begin{itemize}
	\item Operating system needs its own resources to make decisions
	\item A layered structure between request and resources
	\end{itemize}
\textbf{Detail Abstraction: Sharing}
	\begin{itemize}
	\item Device context and method calling
	\item Unifed interface for application devices
	\item Exclusive access \- 1 process on a constrained computer (virtual memory)
	\end{itemize}
Different types of OS open up choices in scheduling algorithms, etc depending on load, tasks, resources
	\begin{itemize}
	\item Mainframes
	\item Realtime -- Has dealines on tasks
		\begin{itemize}
		\item \textbf{Soft} - (Can miss dealines) A dvd player and its FPS
		\item \textbf{Hard} - if a deadline is missed, might as well have not tried (nuclear plant auto-pilot)
		\end{itemize}
	\item Embedded -- A Car, A linking library to abstract I/O
	\item Server -- Linux is still a server class OS
	\end{itemize}
We will look at medium sized systesm since they are constrained enough so we can't be naive, but they have enough resources so that we can share.
In computer science, often the same problem will be solved, historically, twice. This is frequently the result of \textbf{Paradigm shifts}\\\\ 
\textbf{Remember}: Computer time is expensive\\\\
\textbf{Von Neumann Architecture} - data and code occupied a unifed memory.\\
We will start with so few resources that we can't share.
\end{document}
